\documentclass[twoside,colorback,accentcolor=tud4c,11pt]{tudreport}

\usepackage{ngerman}
\usepackage[utf8]{inputenc} 
\usepackage[T1]{fontenc}
\usepackage{cancel}
\usepackage{mathtools}
\usepackage{float}
\usepackage{hyperref}
\usepackage{isotope}
\usepackage[deletedmarkup=xout]{changes}

\title{VERSUCH 5.4: Tieftemperatur Messung an Suprafluidem Helium }

\subtitle{
\begin{tabular}{p{4cm}ll} 
 Name & Dominik Pfeiffer   &   Jonas Fischer\\
 Matrikelnummer & 2913632  & 2240758 \\
 E-mail& \textaccent{dominik@diepfeiffers.de} & \textaccent{jonas.fischer.42@gmail.com}\\
 \\Versuchsbetreuung & • \\
 Durchführung& 29.05.2017 \\
 Abgabetermin& 19.06.2017
 \end{tabular}}
\institution{Institut für Festkörperphysik}
\sponsor{Wir erklären hiermit, dass die vorliegende Arbeit eigenständig, ohne fremde Hilfe und mit der angegebenen Literatur erarbeitet wurde. Alle Passagen aus Literatur und Internet sind als solche gekennzeichnet. Diese Arbeit liegt weder in gleicher noch ähnlicher Weise einer Prüfungskommission vor.\\\\ 
\begin{tabular}{lp{2em}lp{2em}l}
 \hspace{4cm}   && \hspace{4cm}  && \hspace{4cm}
 \\\cline{1-1}\cline{3-3}\cline{5-5}
    Darmstadt, den \today && Dominik Pfeiffer && Jonas Fischer 
\end{tabular}  
 }   
\begin{document}

\maketitle 

\tableofcontents

\chapter{Ziel des Versuchs}
•
\chapter{Physikalische Grundlagen}
\section{Kühlverfahren zum Erreichen tiefer Temperaturen}
\subsection{Verflüssigung}
\subsection{Verdampfungskühlung}
\subsection{Adiabatische Entmagnetisierung}
\subsection{\isotope[3]{He}-\isotope[4]{He} Entmischungskühlung}
\section{Thermometrie bei tiefen Temperaturen}
\subsection{Primär}
\subsection{Sekundär}
\section{Curie-/Curie-Weiss-Gesetz}
\subsection{Herleitung des Curie-Gesetztes}
Für ein Atom mit dem Gesamtdrehmoment $\vec{J}$ im äußeren Feld $\vec{B}$ schreibt sich die potentielle Energie als
\begin{equation}
V=\mu_{b} g_{J}\vec{J}\cdot\vec{B}
\end{equation}
dabei gilt $\vec{J}\cdot\vec{B} =m_{z}B_{z}$ mit $m_{z}\in\left[-J,J\right]$. Aus der Boltzmann-Statistik erhält man die Zustandssumme der J+1 möglichen Energiezustände als
\begin{equation}
Z=\sum_{m_{J}=-J}^{J} e^{-m_{z}\cdot\frac{\mu_{b}g_{J}B_{z}}{k_{B}T}}=\sum_{m_{J}=-J}^{J} e^{-m_{z}\cdot\frac{\xi}{J}}\,mit\, \xi = \frac{\mu_{b}g_{J}B_{z}\cdot J}{k_{B}T}.
\end{equation}
Daraus folgt die Darstellung der Zustandssumme über die Hyperbelfunktion sinh zu
\begin{equation}
Z=\frac{\sinh\left(\frac{2J+1}{2J}\cdot\xi\right)}{\sinh\left(\frac{\xi}{2J}\right)}.
\end{equation}
Mit der freien Energie $F=U-TS=-Nk_{B}T\ln(Z)$ ergibt sich für die Magnetisierung M eines idealen Paramagneten folgender Zusammenhang:
\begin{equation}
M=-\frac{1}{V}\frac{\partial F}{\partial B}=-\frac{1}{V}\frac{\partial F}{\partial Z}\frac{\partial Z}{\partial \xi}\frac{\partial \xi}{\partial B}
\end{equation}
und mit den partiellen Ableitungen
\begin{align}
\frac{\partial F}{\partial Z}=&\frac{-Nk_{B}T}{Z}=-Nk_{B}T\cdot\frac{\sinh\left(\frac{\xi}{2J}\right)}{\sinh\left(\frac{2J+1}{2J}\cdot\xi\right)}\\
\frac{\partial Z}{\partial \xi}=&\frac{(2J+1)\sinh\left(\frac{\xi}{2J}\right)\cosh\left(\frac{2J+1}{2J}\cdot\xi\right)-\cosh\left(\frac{\xi}{2J}\right)\sinh\left(\frac{2J+1}{2J}\cdot\xi\right)}{2J\cdot\sinh^2\left(\frac{\xi}{2J}\right)}\\
\frac{\partial \xi}{\partial B}=&\frac{\mu_{b}g_{J}\cdot J}{k_{B}T}
\end{align}
folgt damit für M
\begin{align}
M=&\frac{1}{V}\cdot\frac{\mu_{b}g_{J}\cdot J}{\deleted{ k_{B}T}}\cdot N \deleted{k_{B}T}\cdot\frac{\deleted{\sinh\left(\frac{\xi}{2J}\right)}}{\sinh\left(\frac{2J+1}{2J}\cdot\xi\right)}\cdot\frac{(2J+1)\sinh\left(\frac{\xi}{2J}\right)\cosh\left(\frac{2J+1}{2J}\cdot\xi\right)-\cosh\left(\frac{\xi}{2J}\right)\sinh\left(\frac{2J+1}{2J}\cdot\xi\right)}{2J\cdot\sinh^{\deleted{2}}\left(\frac{\xi}{2J}\right)}\\
=&\frac{N\mu_{b}g_{J}\cdot J}{V}\left[\frac{2J+1}{2J}\cdot\coth\left(\frac{2J+1}{2J}\xi\right)-\frac{1}{2J}\cdot\coth\left(\frac{\xi}{2J}\right)\right]
\end{align}
wobei der Teil in de eckigen Klammern der Brillouin-Funktion $B_{J}(x)$ entspricht. Verwendet man nun die Reihendarstellung $\coth (x)=\frac{1}{x}+\frac{x}{3}+O(x^3)$ für kleine x, was dem Grenzwert $\lim_{\xi\to0}\xi=\lim_{T\to \infty}\xi (T)$ entspricht, so erhält man aus der obigen Darstellung
\begin{align}
M=&\frac{N\mu_{b}g_{J}\cdot \deleted{J}}{V}\cdot\frac{J+1}{\deleted{J}}\frac{\xi}{3}\\
=&\frac{N\mu_{B}g_{J}}{V}\frac{J+1}{3}\frac{\mu_{B}g_{J}B_{z}J}{k_{B}T}\\
=&\frac{N\left(\mu_{B}g_{J}\right)^2B_{z}}{3Vk_{B}}\frac{J(J+1)}{T}.
\end{align}
Mit der Beziehung $\chi=\mu_{0}\frac{\partial M}{\partial B}$ findet man schließlich
\begin{equation}
\chi=\mu_{0}\frac{\partial M}{\partial B}=\underbrace{\frac{\mu_{0}N\left(\mu_{B}g_{J}\right)^2}{3Vk_{B}}J(J+1)}_{\equiv C}\cdot\frac{1}{T}=\frac{C}{T}
\end{equation}
das bekannte Curie-Gesetz mit der Curie-Konstanten $C=\frac{\mu_{0}N\left(\mu_{B}g_{J}\right)^2}{3Vk_{B}}J(J+1)$.
\section{Eigenschaften von \isotope[3]{He} und \isotope[4]{He}}
\chapter{Versuchsdurchführung und Auswertung}
•
\chapter{Fazit}	
•
\renewcommand{\bibname}{Literatur}
\begin{thebibliography}{0}
\bibitem {refname} •

\end{thebibliography}
\end{document}    
